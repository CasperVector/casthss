% Copyright (c) 2008-2009 solvethis
% Copyright (c) 2010-2018 Casper Ti. Vector
% Public domain.
%
% 使用前请先仔细阅读 biblatex-caspervector 的文档,
% 特别是其中的 FAQ 部分和用红色强调的部分。
\documentclass[UTF8]{casthss}

% 使用 biblatex 排版参考文献,并规定其格式(详见 biblatex-caspervector 的文档)。
% 这里按照西文文献在前,中文文献在后排序(“sorting = ecnyt”);
% 若需按照中文文献在前,西文文献在后排序,请设置“sorting = cenyt”;
% 若需按照引用顺序排序,请设置“sorting = none”。
% 若需在排序中实现更复杂的需求,请参考 biblatex-caspervector 的文档。
\usepackage[backend = biber, style = caspervector, utf8, sorting = ecnyt]{biblatex}

% 按学校要求设定参考文献列表的字体和段间距。
\renewcommand*{\bibfont}{\zihao{5}}
\setlength{\bibitemsep}{0pt}

% 设定文档的基本信息。
\casthssinfo{
	cthesisname = {博士学位论文}, ethesisname = {dissertation},
	ctitle = {测试文档}, etitle = {Test Document},
	cauthor = {某某}, eauthor = {Test},
	cdate = {某年某月}, edate = {Month, Year},
	cmentor = {某某\ 研究员\\(中国科学院某某研究所)},
	ementor = {Prof.\ Somebody},
	cdegree = {理学博士}, edegree = {Doctor of Philosophy},
	cmajor = {某某专业}, emajor = {Some Major},
	cschool = {中国科学院某某研究所},
	eschool = {Institute of Something, Chinese Academy of Sciences},
	ckeywords = {其一,其二}, ekeywords = {First, Second},
	% 导师多于一位时可通过 mentorlines 字段调整封面“指导教师”部分的行数。
	% 涉密论文可将 confidential 字段设为“秘密\ ★\ 10 年”等内容。
	mentorlines = {2}, confidential = {}
}
% 载入参考文献数据库(注意不要省略“.bib”)。
\addbibresource{thesis.bib}

% 普通用户可删除此段,并相应地删除 chap/*.tex 中的
% “\casthssffaq % 中文测试文字。”一行。
\usepackage{color}
\def\casthssffaq{%
	\emph{\textcolor{red}{casthss 文档模版最常见问题:}}

	\texttt{\string\cite}、\texttt{\string\parencite} %
	和 \texttt{\string\supercite} 三个命令分别产生%
	未格式化的、带方括号的和上标且带方括号的引用标记:%
	\cite{test-en},\parencite{test-zh}、\supercite{test-en, test-zh}。%
}

\begin{document}
	% 以下为正文之前的部分,默认不进行章节编号。
	\frontmatter
	% 此后到下一 \pagestyle 命令之前不排版页眉或页脚。
	\pagestyle{empty}
	% 自动生成中文封面。
	\maketitle
	% 自动生成西文封面。
	\cleardoublepage
	\emaketitle
	% 原创性声明和使用授权说明。
	\cleardoublepage
	% Copyright (c) 2018 Casper Ti. Vector
% Public domain.

\vspace*{-4em}\vspace*{\fill}
\pdfbookmark[0]{中国科学院大学学位论文原创性声明和授权使用说明}{origin}
\begin{center}%
	\heiti\bfseries\zihao{4}
	中国科学院大学\\研究生学位论文原创性声明%
\end{center}

本人郑重声明:
所呈交的学位论文是本人在导师的指导下独立进行研究工作所取得的成果。
尽我所知,除文中已经注明引用的内容外,
本论文不包含任何其他个人或集体已经发表或撰写过的研究成果。
对论文所涉及的研究工作做出贡献的其他个人和集体,
均已在文中以明确方式标明或致谢。

\begin{flushright}
	作者签名:\hspace{8em}\mbox{}\\
	日期:\hspace{2\ccwd}\hspace{8em}\mbox{}%
\end{flushright}
\vspace*{\stretch{0.5}}

\begin{center}
	\heiti\bfseries\zihao{4}
	中国科学院大学\\学位论文授权使用声明%
\end{center}

本人完全了解并同意遵守中国科学院有关保存和使用学位论文的规定,
即中国科学院有权保留送交学位论文的副本,允许该论文被查阅,
可以按照学术研究公开原则和保护知识产权的原则公布该论文的全部或部分内容,
可以采用影印、缩印或其他复制手段保存、汇编本学位论文。\par
涉密及延迟公开的学位论文在解密或延迟期后适用本声明。

\begin{flushright}
	作者签名:\hspace{8em}导师签名:\hspace{8em}\mbox{}\\
	日期:\hspace{2\ccwd}\hspace{8em}%
	日期:\hspace{2\ccwd}\hspace{8em}\mbox{}\\
\end{flushright}
\vspace*{\fill}

% vim:ts=4:sw=4


	% 此后到下一 \pagestyle 命令之前正常排版页眉,居中排版页脚。
	\cleardoublepage
	\fancypagestyle{plain}{\fancyfoot[C]{\zihao{-5}\normalfont{\thepage}}}
	\pagestyle{plain}
	% 重置页码计数器,用大写罗马数字排版此部分页码。
	\setcounter{page}{0}
	\pagenumbering{Roman}
	% 中西文摘要。
	% Copyright (c) 2014,2016 Casper Ti. Vector
% Public domain.

\begin{cabstract}
	\casthssffaq % 中文测试文字。
\end{cabstract}

\begin{eabstract}
	Test of the English abstract.
\end{eabstract}

% vim:ts=4:sw=4

	% 自动生成目录。
	\tableofcontents

	% 以下为正文部分,页码在页面外边界排版,默认要进行章节编号。
	\mainmatter
	\fancypagestyle{plain}%
		{\fancyfoot{}\fancyfoot[LE,RO]{\zihao{-5}\normalfont{\thepage}}}
	\pagestyle{plain}
	% 各章节。
	% Copyright (c) 2014,2016,2018 Casper Ti. Vector
% Public domain.

\chapter{引言}
\casthssffaq % 中文测试文字。

% vim:ts=4:sw=4

	% Copyright (c) 2014,2016 Casper Ti. Vector
% Public domain.

\chapter{章节}
\casthssffaq % 中文测试文字。

% vim:ts=4:sw=4

	% Copyright (c) 2014,2016,2018 Casper Ti. Vector
% Public domain.

\chapter{结论和展望}
\casthssffaq % 中文测试文字。

% vim:ts=4:sw=4


	% 正文中的附录部分。
	\appendix
	% 排版参考文献列表。bibintoc 选项使“参考文献”出现在目录中;
	\printbibliography[heading = bibintoc]
	% 各附录。
	% Copyright (c) 2014,2016 Casper Ti. Vector
% Public domain.

\chapter{附件}
\casthssffaq % 中文测试文字。

% vim:ts=4:sw=4


	% 以下为正文之后的部分,默认不进行章节编号。
	\backmatter
	% 致谢。
	% Copyright (c) 2014,2016 Casper Ti. Vector
% Public domain.

\chapter{致谢}
\casthssffaq % 中文测试文字。

% vim:ts=4:sw=4

	% 简历及成果。
	% Copyright (c) 2018 Casper Ti. Vector
% Public domain.

\chapter{作者简历和研究成果目录}

\section*{作者简历}
\begin{itemize}[leftmargin = \parindent]
\item 2008 年 9 月 -- 2009 年 8 月在某某大学某学部某学院学习。
\item 2009 年 9 月 -- 2013 年 8 月在某某大学某某与某某某某学院获得学士学位。
\item 2013 年 9 月 -- 2014 年 8 月在家报考中国科学院某某研究所。
\item 2014 年 9 月至今在中国科学院某某研究所攻读博士学位。
\end{itemize}

\section*{已发表或正式接受的学术论文}
% \defbibenvironment、\nocite、refsection、refcontext 的用法参考 biblatex 文档。
\defbibenvironment{publications}%
	{\begin{itemize}[leftmargin = \parindent]}{\end{itemize}}{\item}
\begin{refsection}
\renewcommand*{\bibfont}{}
\nocite{myself}
\begin{refcontext}[sorting = none]
\printbibliography[heading = none, env = publications]
\end{refcontext}
\end{refsection}

% 无专利时不必列出。
% \section*{申请或已获得的专利}

\section*{参加的研究项目及获奖情况}
\begin{itemize}[leftmargin = \parindent]
\item 参与某某研究项目。
\end{itemize}

% vim:ts=4:sw=4

\end{document}

% vim:ts=4:sw=4
